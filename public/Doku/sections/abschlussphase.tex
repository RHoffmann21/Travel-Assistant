\subsection{Soll/Ist-Vergleich}
\label{sec:Abschlussphase:Soll/Ist-Vergleich}

Das Projekt wurde wie geplant umgesetzt. Allerdings konnte der Zeitplan nicht vollständig eingehalten werden:

In der Projektdefinitionsphase konnte die eingeplante Zeit eingehalten und sogar verkürzt werden. Die Ist-Analyse fiel aufgrund diverser in der Vergangenheit liegender Berührungspunkte relativ einfach aus.

Die Planungsphase des Projekts wurde grundsätzlich eingehalten, jedoch gab es kleinere Schwankungen, die innerhalb dieser Phase durch andere ausgeglichen wurden. Für das Erstellen des Anwendungsfalldiagramms und der Sketche musste weniger Zeit aufgewendet werden, da im schulischen Kontext eine Auffrischung stattfand. Aufgrund von Speicherproblemen während der Entwicklung musste die Datenbank mehrmals angepasst werden.

Die Projektdurchführungsphase nimmt den Großteil der Stunden in Anspruch. Diese Phase benötigte mehr Zeit als ursprünglich geplant. Für die Erstellung der Docker-Umgebung und der Grundstruktur der Webanwendung wurde weniger Zeit benötigt, da dies aufgrund der regelmäßigen Anwendung in anderen Projekten schneller umgesetzt werden konnte als ursprünglich geplant. Bei der Berechnungslogik konnte Zeit eingespart werden. Der Grund dafür liegt in wegfallenden Berechnungen/Feldern, die nicht mehr benötigt werden. Dadurch ist auch der Aufwand für die Tests dieser kritischen Funktionen geringer geworden. Der Aufwand für die Erstellung und Strukturierung der Fragen wurde jedoch unterschätzt. Um den Überblick über die Abfolge der Fragen zu behalten, musste ein Decision Tree (im Anhang unter \sct{sec:Anhang:DecisionTree}{Decision Tree} zu finden) erstellt werden, um nachvollziehen zu können, an welcher Stelle sich der Code/Chat befindet und zu welcher Abfolge von Fragen dieser führt. Bei der Implementierung der Mobil-First-Oberflächen hat die Verwendung des Frameworks \gl{React} anfangs Schwierigkeiten bereitet und zusätzliche Recherche erfordert. Auch die Integration des \gl{IDP} hat mehr Zeit in Anspruch genommen, da der Autor bisher nur Server-seitige Anwendungen mit dem \gl{IDP} verbunden hat. 

In der Projektabschlussphase konnte wiederum Zeit eingespart werden, da die Abnahme nur aus einer Präsentation des Tools und einem Ausblick bestand, da die Softwarelösung in der ersten Version noch nicht produktiv einsetzbar ist. Auch bei der internen Dokumentation wurde Zeit eingespart, da parallel zum Erstellen von Funktionen eine entsprechende Dokumentation mithilfe von \gl{JSdoc} erstellt wurde. Der Zeitaufwand für die Erstellung der Projekt-Dokumentation ist höher als geplant gewesen, aufgrund ihres Umfangs.

Eine Übersicht über die geplanten und tatsächlich benötigten Stunden für die einzelnen Phasen und Arbeitspakete ist im Anhang unter \sct{sec:Anhang:Soll-Ist-Vergleich}{Soll-Ist-Zeitplan} zu finden.

\subsection{Lessons Learned}
\label{sec:Abschlussphase:Lessons Learned}

Im Laufe des Projekts gab es, wie auch zuvor im \sct{sec:Abschlussphase:Soll/Ist-Vergleich}{Soll/Ist-Vergleich} beschrieben, einige Herausforderungen, welche jedoch überwunden werden konnten. Die Erstellung dieser Softwarelösung und die einhergehenden Lerneffekte haben sich als sehr wertvoll erwiesen. So konnten Fähigkeiten im Bereich der Frontendentwicklung deutlich gefestigt und erweitert werden, da der Fokus im sonstigen Arbeitsalltag auf der Backend-Entwicklung liegt. Planung ist sehr wichtig, das hat sich beim Erstellen der Softwarelösung deutlich bemerkbar gemacht. Das Projekt verlief ohne große Hürden, da in vielen Phasen auf die Planung der vorangegangenen Phasen zurückgegriffen werden konnte.

\subsection{Ausblick}
\label{sec:Abschlussphase:Ausblick}
Wie bereits erwähnt, ist die Softwarelösung in seiner jetzigen Form noch nicht produktiv einsetzbar. Es werden noch einige Funktionalitäten benötigt:

\begin{itemize}
    \item Englische Übersetzung für nicht Deutschsprachige Kollegen
    \item Benachrichtigungen per E-Mail (z.B. Freigaben oder Status Änderungen)
    \item Konfiguration von Pauschalen
    \item Logging von Benutzer Aktionen
\end{itemize}
