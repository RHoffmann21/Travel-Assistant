\subsection{Soll/Ist-Vergleich}
\label{sec:Abschlussphase:Soll/Ist-Vergleich}

Das Projekt wurde wie geplant umgesetzt. Allerdings konnte der Zeitplan nicht vollständig eingehalten werden:

In der Projektdefinitionsphase konnte die eingeplante Zeit eingehalten und sogar verkürzt werden. Die Ist-Analyse fiel aufgrund diverser in der Vergangenheit liegender Berührungspunkte relativ einfach aus.

Die Planungsphase des Projekts wurde grundsätzlich eingehalten, jedoch gab es kleinere Schwankungen, die innerhalb dieser Phase durch andere ausgeglichen wurden. Für das Erstellen des Anwendungsfalldiagramms und der Sketche musste weniger Zeit aufgewendet werden, da im schulischen Kontext eine Auffrischung stattfand. Aufgrund von Speicherproblemen während der Entwicklung musste die Datenbank mehrmals angepasst werden.

Die Projektdurchführungsphase nimmt den Großteil der Stunden in Anspruch. Diese Phase hat mehr Zeit benötigt als ursprünglich geplant. Für die Erstellung der Docker-Umgebung und der Grundstruktur der Webanwendung wurde weniger Zeit benötigt, da dies aufgrund der regelmäßigen Anwendung in anderen Projekten schneller umgesetzt werden konnte als ursprünglich geplant. Bei der Berechnungslogik konnte massiv Zeit eingespart werden. Der Grund dafür liegt in wegfallenden Berechnungen, die nicht mehr benötigt werden. Dadurch ist auch der Aufwand für die Qualitätssicherung und den Tests dieser kritischen Funktionen geringer geworden. Der Aufwand für die Erstellung und Strukturierung der Fragen wurde jedoch unterschätzt. Um den Überblick über die Abfolge der Fragen zu behalten, musste ein Decision Tree erstellt werden (Im Anhang unter Decision Tree zu finden \todo), um nachvollziehen zu können, an welcher Stelle sich der Code/Chat befindet und zu welcher Abfolge von Fragen dieser führt. Bei der Implementierung der Mobil-First-Oberflächen hat die Verwendung des Frameworks \gl{React} anfangs Schwierigkeiten bereitet und zusätzliche Recherche erfordert. Auch die Integration des \gl{IDP} hat mehr Zeit in Anspruch genommen, da der Autor bisher nur Server-Side-Applikationen mit dem \gl{IDP} verbunden hat. 

In der Projektabschlussphase konnte wiederum Zeit gewonnen werden, da die Abnahme nur aus einer Präsentation des Tools und einem Ausblick bestand, da die Softwarelösung in der ersten Version noch nicht produktiv einsetzbar ist. Auch bei der internen Dokumentation wurde Zeit eingespart, da parallel zum Erstellen von Funktionen eine entsprechende Dokumentation mithilfe von \gl{JSdoc} erstellt wurde. Die Zeit für die Erstellung der Projekt-Dokumentation ist höher als geplant, aufgrund ihres Umfangs.

Eine Detailierter Vergleich zu den geplanten und tatsächlich benötigten Stunden für die jeweiligen Phasen und Arbeitspakete ist im Anhang unter Soll-/Ist Stunden Vergleich zu finden. \todo

\subsection{Lessons Learned}
\label{sec:Abschlussphase:Lessons Learned}
\todo umschreiben
Im Laufe des Projekts gab es einige Herausforderungen, die jedoch gemeistert wurden und zu einem Lerneffekt führten. Außerdem konnten die Fähigkeiten im Bereich der Frontendentwicklung ausgebaut werden, während der Fokus normalerweise auf der Backendentwicklung liegt. Eine wichtige Erkenntnis ist, dass es manchmal besser ist, eine einfachere Lösung zu wählen, anstatt eine vermeintlich schönere, aber wesentlich komplexere. Dadurch wird der Code leichter verständlich und nachvollziehbar.

\subsection{Ausblick}
\label{sec:Abschlussphase:Ausblick}
Wie bereits erwähnt, ist die Softwarelösung in seiner jetzigen Form noch nicht produktiv einsetzbar. Es werden noch einige Funktionalitäten benötigt:

\begin{itemize}
    \item Englische Übersetzung für nicht Deutschsprachige Kollegen
    \item Benachrichtigungen per E-Mail (z.B. Freigaben oder Status Änderungen)
    \item Konfiguration von Pauschalen
    \item Logging von Benutzer Aktionen
\end{itemize}
