\subsection{Projektphasen und Zeitplanung}

Wie von der IHK Ostwestfalen zu Bielefeld vorgegeben, stehen für die Durchführung dieses Projektes 80 Stunden zur Verfügung. Dazu wurden grobe Arbeitspakete definiert und der Zeitaufwand für deren Bearbeitung geschätzt. Eine Übersicht über die Arbeitspakete und die ursprüngliche Schätzung des Zeitaufwandes (und der tatsächlich benötigten Zeit) ist im Anhang unter Zeitplan \sct{sec:Anhang:sollistvergleich}{Soll-/Ist-Zeitplanung} zu finden.


\subsection{Ressourcenplanung}

Eine Übersicht über die während der Projektlaufzeit voraussichtlich benötigten Ressourcen findet sich im Anhang unter \sct{sec:Anhang:VerwendeteRessourcen}{Verwendete Ressourcen}. Diese Übersicht wurde während der Projektlaufzeit laufend aktualisiert, so dass auch während der Projektlaufzeit neu hinzukommende Ressourcen aufgeführt sind.
Bei der Auswahl der Software wurde darauf geachtet, dass bereits Kenntnisse oder Erfahrungen vorhanden waren, um den Zeitaufwand für die Einarbeitung so gering wie möglich zu halten. Darüber hinaus wurden die jeweiligen Lizenz- und Nutzungsbedingungen berücksichtigt. Weitere Aspekte sind unter \sct{sec:Durchführungsphase:AuswahlVonLibraries}{Auswahl von Libraries} zu finden.
