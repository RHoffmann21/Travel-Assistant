Nachdem die aktuelle Situation analysiert wurde, wurde mit der Planung der Applikation begonnen.

\subsection{Applikationsart}

Wie bereits unter \sct{sec:Einführung-Definitionsphase:Projektziel}{Projektziel} genannt, soll die Applikation als Webapplikation entwickelt werden. Ein Grund dafür ist, dass man die Applikation so in der Zukunft
vergleichsweise einfach aktualisieren kann. Der Technologie Stack \gl{MERN} wirf für die Erstellung der Softwarelösung gewählt, weil diese generell Etabliert und bereits in dem Software Development Team verwendet wird und in diverse Projekten bereits umgesetzt wird. Dadurch kann gegebenenfalls vom Know-How anderer Personen im Team profitiert werden. So wird die Serverapplikation mit \gl{express} auf der Laufzeitumgebung \gl{nodejs} entwickelt werden. Die \gl{SPA} mit \gl{React} entwickelt werden und \gl{MongoDB} zur Speicherung von Daten verwendet werden.

\subsection{Applikationsarchitektur}
\label{sec:Planungsphase:Applikationsarchitektur}

Die Softwarelösung wird in zwei Teilapplikationen entwickelt: \sct{sec:Planungsphase:Frontend}{Frontend} und \sct{sec:Planungsphase:Backend}{Backend}. Eine Backend Applikation ist Notwenig da sich bereits auf eine \gl{SPA} festgelegt wurde. Da eine \gl{SPA} nicht sicher mit Daten interagieren kann.

\subsubsection{Frontend}
\label{sec:Planungsphase:Frontend}

Das Frontend wird durch eine \gl{SPA} repräsentiert. Eine \gl{SPA} ist eine Applikation, die vollständig im Browser ausgeführt wird.  Im Gegensatz zu normalen Webanwendungen wird nicht bei jeder Interaktion mit dem Server \gl{HTML} zurückgegeben und vom Browser dargestellt.Stattdessen wird die Interaktion durch browserseitiges \gl{JS} ausgeführt.

\subsubsection{Backend}
\label{sec:Planungsphase:Backend}

Das Backend läuft auf dem Server und und übernimmt die gesamte Datenverwaltung und Businesslogik. Mit \gl{mongoose} kann das Backend dann auf die gespeicherten Daten der \gl{MongoDB} zugreifen.

Da die beiden Teilapplikationen miteinander kommunizieren müssen stellt das Backend eine \gl{HTTP}-\gl{API} zur Verfügung. Über diese kann dann das Frontend Informationen senden. Um so z.B. \glp{RA} einreichen zu können.
Das Backend wird auf Basis der \gl{MSC}-Architektur entwickelt. Dabei werden die Komponenten nach Typ unterteilt. In der \gl{MSC}-Architektur gibt es folgende Komponententypen:

\begin{enumerate}
  \item \textbf{M}odel\\
  Das Model beschreibt die zu speichernden Daten im Rahmen einer einzelnen Entität. So werden im Model die klassischen CRUD-Operationen durchgeführt. Wie das Kreieren einer \gl{RA}.
  \item \textbf{S}ervice\\
  Im Service findet die gesamte Logik statt. Services sind die einzigen Komponenten der Applikation, die mit Models interagieren. Sie bieten die Möglichkeit, die CRUD-, und gegebenenfalls noch weitere, Operationen auszuführen. Wie die Hintergrundberechnungen von Pauschalen.
  \item \textbf{C}ontroller\\
  Controller sind die Komponenten, die die tatsächliche Interaktion mit der Außenwelt, also dem \gl{HTTP}-Client bestreiten. Sie interagieren mit den Services, um die Anfragen des Clients auszuführen. Der Punkt, an dem die Daten vom Frontend an die Services zur Verarbeitung gesendet werden.
\end{enumerate}

Neben diesen Hauptkomponenten gibt es die folgenden Komponenten:

\begin{itemize}
  \item Middleware\\
Neben den vorinstallierten oder heruntergeladenen Middlewares können auch eigene erstellt werden. Im konkreten Fall wird eine Middleware verwendet, die den Zugriff auf bestimmte Bereiche der Applikation einschränkt. Dadurch wird eine Anfrage bereits vor dem Eintreffen im Controller beantwortet, falls bestimmte Anforderungen nicht erfüllt sind. So können z.B. nur Benutzer mit der Prüfer Rolle die Prüfungsansicht einsehen.
  \item Tests\\
  Die Tests dienen dazu, die einzelnen Komponenten auf verschiedene erwartete Ergebnisse hin zu überprüfen. Dabei werden die kritischen Berechnungsfunktionen auf ihr vorgesehenes Verhalten geprüft.
\end{itemize}

\subsection{Models und Datenstruktur}

Die Anforderungen machen deutlich, dass mehrere Modelle erforderlich sind. So wird ein Modell für die Länder und deren Pauschalen benötigt. Die Reisekostenabrechnung selbst, die sich aus dem Chatverlauf ergibt, sowie Grundinformationen und Dienstreisen welche die Informationen aus dem Chat verarbeitet. Sowie ein Modell, das einige Berechnungsgrundlagen speichert. Außerdem wird ein Benutzermodell benötigt, um die jeweiligen Vorgesetzten zu speichern.

\subsection{Feststellung der Benutzerberechtigungen}
\label{sec:Planungsphase:Benutzerberechtigungen}

Das unter \sct{sec:Analysephase:Benutzerklassifizierung}{Benutzerklassifizierung} beschriebene Benutzerkonzept sieht drei Benutzerklassen vor. Es wird davon ausgegangen, dass jeder Benutzer, der sich an der Anwendung bzw. am \gl{IDP} anmeldet, zunächst ein Regulärer Benutzer ist. Anschließend wird geprüft, ob ein Benutzer über eine Rolle verfügt, die ihm über den \gl{IDP} zugewiesen wurde.
