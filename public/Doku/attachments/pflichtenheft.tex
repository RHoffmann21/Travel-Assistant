\subsection{Pflichtenheft}
\label{sec:Anhang:Pflichtenheft}
\textit{Version 1.1}\\

\textbf{Projektbeschreibung}\\
Das Projekt soll die erste Version einer Softwarelösung sein, die in Zukunft die aktuelle Methode der Erstellung von Reisekosten mit einer Microsoft Excel Datei, ablösen wird.

\textbf{Anforderungen}\\
Die nicht funktionalen Anforderungen, die im Lastenheft niedergeschrieben sind, werden unverändert übernommen. Die funktionalen und Anforderungen werden wie folgt ergänzt:
\begin{itemize}
\item Nutzer sollen zu Beginn ihren Vorgesetzten auswählen können
\item Nutzer sollen ihre Vorgesetzten im Nachhinein ändern können
\end{itemize}

\textbf{Architektur}\\
Die Softwarelösung wird mit der Full-Stack Lösung MERN entwickelt.
Somit wird die Softwarelösung in zwei Abschnitte unterteilt: das Frontend, welches React nutzt, und das Backend, welches Express auf einem Node.js Server nutzt. Die Daten werden in einer MongoDB gespeichert.
Alle Komponenten laufen in einzelnen Docker Containern.
Die Softwarelösung wird mit JavaScript programmiert.

\textbf{Schnittstellen}\\
Außer den Schnittstellen, die das Frontend und das Backend benötigen, um miteinander zu kommunizieren, wird eine weitere Schnittstelle zum hauseigenen IdP benötigt, um Login, Rechte und Benutzerverwaltung zu übernehmen.

\textbf{Benutzeroberfläche}\\
Die Benutzeroberfläche wird für bestimmte Bereiche wie folgt umgesetzt:\\
Mobile-First:
\begin{itemize}
\item Reisekostenabrechnung erstellen
\item Reisekostenabrechnung bearbeiten
\item Übersicht erstellter Reisekostenabrechnungen
\end{itemize}
Desktop-Ansicht:\\
\begin{itemize}
\item Reisekostenabrechnung genehmigen
\item Reisekostenabrechnung prüfen
\item Übersicht freigegebener Reisekostenabrechnung 
\item Übersicht genehmigter Reisekostenabrechnung
\end{itemize}

Ein grobes Layout der Oberflächen sind im Anhang unter \sct{sec:Anhang:Sketches}{Sketches} zu finden.
Das Corporate Design wird überwiegend eingesetzt. Nichtsdestotrotz, werden vereinzelte Komponenten der Softwarelösung abweichen, da diese bis dato einzigartig sind und dementsprechend nur an das Coporate Design angelehnt werden können.

\textbf{Benutzerrollen und -berechtigungen}\\
Die Benutzerrollen werden wie im Lastenheft beschrieben übernommen und wie folgt erweitert:
\begin{itemize}
\item Nutzer – Mitarbeiter im Außendienst
\item Prüfer – Mitarbeiter der Accounting Abteilung
\item GF – Geschäftsführung
\item Vorgesetzter - Übergeordneter Mitarbeiter des Außendienst Mitarbeiters
\end{itemize}

Nutzer können Reisekostenabrechnungen erstellen, einsehen und freigeben.
Vorgesetzte und die Geschäftsführung dürfen Reisekostenabrechnungen ihrer Unterstellten einsehen und genehmigen.
Prüfer können genehmigte Reisekostenabrechnungen prüfen und dementsprechend bestätigen oder mit Kommentar ablehnen.
Die Geschäftsführung erbt die Berechtigung vom Vorgesetzten, welcher wiederum vom Nutzer erbt. Prüfer erben auch vom Nutzer.
Ein Anwendungsfalldiagramm für diese "Use-Cases" kann im Anhang unter \sct{sec:Anhang:Anwendungsfalldiagramm}{Anwendungsfalldiagramm} eingesehen werden.

\pagebreak

\textbf{Status}\\
Die möglichen Status einer RA werden wie folgt umgesetzt:
\begin{itemize}
\item \verb|pending| - Noch nicht zur Prüfung/Freigabe gesendete Reisekostenabrechnung
\item \verb|verified| - Von Vorgesetzten/GF freigegeben Reisekostenabrechnung
\item \verb|accepted| - Von den Prüfern akzeptierte Reisekostenabrechnung
\item \verb|declined| - Von den Vorgesetzten/GF/Prüfer abgelehnte Reisekostenabrechnung
\item \verb|needsEditing| - Von den Vorgesetzten/GF/Prüfer zur Anpassung zurückgegebene Reisekostenabrechnung
\end{itemize}

\textbf{Interne Dokumentation}\\
Die interne Dokumentation wird durch Kommentierung im Quellcode realisiert.

\textbf{Abnahme}\\
Die Abnahme- und Testkriterien werden wie im Lastenheft beschrieben übernommen. Es findet keine gewöhnliche Projektabnahme statt. Da diese erste Version der Softwarelösung noch nicht vollumfänglich eingesetzt werden kann, ist eine Präsentation dieser an die Accounting-Abteilung ausreichend, in welcher die Anforderungen in der Softwarelösung wiedergegeben werden.
  
