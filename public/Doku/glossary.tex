\newglossaryentry{RA}{
  name={RA},
  description={Reisekostenabrechnung},
  first={Reisekostenabrechnung (\glsentrytext{RA})},
  plural={RAs},
  descriptionplural={Reisekostenabrechnungen},
  firstplural={Reisekostenabrechnungen (\glsentryplural{RA})}
}
\newglossaryentry{MED}{
  name={MED},
  description={Microsoft Excel Datei},
  first={Microsoft Excel Datei (MED)},
  plural={MEDs},
  descriptionplural={Microsoft Excel Dateien},
  firstplural={Microsoft Excel Datei (\glsentryplural{MED})}
}
\newglossaryentry{IDP}{
  name={IdP},
  description={Identity Provider},
  first={Identity Provider (\glsentrytext{IDP})}
}
\newglossaryentry{MA}{
  name={MA},
  description={Mitarbeiter},
  first={Mitarbeiter (\glsentrytext{MA})}
}
\newglossaryentry{GF}{
  name={GF},
  description={Geschäftsführung},
  first={Geschäftsführung (\glsentrytext{GF})}
}
\newglossaryentry{HTTP}{
  name={HTTP},
  description={\emph{\textbf{H}yper\textbf{t}ext \textbf{T}ransfer \textbf{P}rotocol}}
}
\newglossaryentry{API}{
  name={API},
  description={\emph{\textbf{A}pplication \textbf{P}rogramming \textbf{I}nterface}}
}
\newglossaryentry{mongoose}{
  name={Mongoose},
  description={\gl{ODM}-Library zur Verbindung mit \gl{MongoDB}, \ext{https://mongoosejs.com}}
}
\newglossaryentry{SPA}{
  name={SPA},
  description={\emph{\textbf{S}ingle \textbf{P}age \textbf{A}pplication}}
}
\newglossaryentry{MongoDB}{
  name={MongoDB},
  description={Dokumentenbasiertes NoSQL-Datenbanksystem, \ext{https://www.mongodb.com}}
}
\newglossaryentry{ODM}{
  name={ODM},
  description={\emph{\textbf{O}bject \textbf{D}ata \textbf{M}odeling}}
}
\newglossaryentry{JS}{
  name={JavaScript},
  description={Skriptsprache, die eine der Haupttechnologien des WWW ist}
}
\newglossaryentry{React}{
  name={React},
  description={Open Source Single Page Application Framework. Umfasst vergleichsweise geringen Funktionsumfang out-of-the-box, ist im Umkehrschluss aber wesentlich flexibler und zum Beispiel im Web und Nativ anwendbar}
}
\newglossaryentry{HTML}{
  name={HTML},
  description={\emph{\textbf{H}yper\textbf{t}ext \textbf{M}arkup \textbf{L}anguage}, \gl{XML}-basierte Beschreibungsnotation für Webumgebungen}
}
\newglossaryentry{XML}{
  name={XML},
  description={\emph{e\textbf{X}tensible \textbf{M}arkup \textbf{L}anguage}}
}
\newglossaryentry{MSC}{
  name={MSC},
  description={\emph{\textbf{M}odel, \textbf{S}ervice, \textbf{C}ontroller}, Architektur zur Gestaltung von Applikationen}
}
\newglossaryentry{nodejs}{
  name={Node.js},
  description={\gl{JS}-Laufzeitumgebung, \ext{https://nodejs.com}}
}
\newglossaryentry{vite}{
  name={vite},
  description={Buildtool für diverse \gl{SPA}-Frameworks, \ext{https://vitejs.dev}}
}
\newglossaryentry{cookie}{
  name={Cookie},
  description={kleine Textdaten, die vom Browser gespeichert und bei \gl{HTTP}-Requests mitgesendet werden}
}
\newglossaryentry{jwt}{
  name={JWT},
  description={\emph{\gl{json} \textbf{W}eb \textbf{T}oken}, \ext{https://jwt.io/introduction}}
}
\newglossaryentry{json}{
  name={JSON},
  description={\emph{\gl{JS} \textbf{O}bject \textbf{N}otation}, Textbasiertes Datenformat zum Datenaustausch zwischen mehreren Anwendungen}
}
\newglossaryentry{CRUD}{
  name={CRUD},
  description={\emph{\textbf{C}reate, \textbf{R}ead, \textbf{U}pdate, \textbf{D}elete}}
}
\newglossaryentry{git}{
  name={Git},
  description={Versionsverwaltungssystem, \ext{https://git-scm.com}}
}
\newglossaryentry{vscode}{
  name={Visual Studio Code},
  description={Entwicklungsumgebung/Texteditor, \ext{https://code.visualstudio.com}}
}
\newglossaryentry{miro}{
  name={miro},
  description={Online-Kollaborationsplattform für u.a. Diagramme, Mindmaps und Unterschiedliche Boards, \ext{https://miro.com/de/}}
}
\newglossaryentry{MERN}{
  name={MERN},
  description={Ein Technologie-Stack, welcher aus den Komponenten: \gl{MongoDB}, \gl{express}, \gl{React} und Node.js. besteht }
}
\newglossaryentry{express}{
  name={Express},
  description={ein serverseitiges Webframework für die JavaScript-basierte Plattform Node.js, \ext{https://expressjs.com} }
}
\newglossaryentry{JSdoc}{
  name={JSdoc},
  description={eine Auszeichnungssprache, die zum Annotieren von JavaScript-Quellcodedateien verwendet wird, \ext{https://jsdoc.app/} }
}
\newglossaryentry{cors}{
  name={CORS},
  description={\textbf{C}ross-\textbf{O}rigin \textbf{R}esource \textbf{S}haring }
}
\newglossaryentry{promise}{
  name={Promise},
  description={\gl{JS}-\gl{API} zum Abhandeln von asynchron ablaufenden Prozessen. engl. Promise: Versprechen, es wird \texttt{await}-ed, dass die Promise eingelöst wird und man so die Daten erhält}
}
\newglossaryentry{css}{
  name={CSS},
  description={\emph{\textbf{C}ascading \textbf{S}tyle\textbf{s}heets}}
}
\newglossaryentry{jest}{
  name={jest},
  description={ein \gl{JS} Testing Framework, \ext{https://jestjs.io/}}
}
\newglossaryentry{REST}{
  name={REST},
  description={\textbf{Re}presentational \textbf{S}tate \textbf{T}ransfer}
}